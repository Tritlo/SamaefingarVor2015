% not needed with polyglossia
\usepackage[utf8]{inputenc}
\usepackage[T1]{fontenc}

%xe/lualatex
%\usepackage{polyglossia}
%\setdefaultlanguage{icelandic}


\usepackage{graphics,amsmath,amsfonts}
\usepackage{amsbsy,amssymb}
\usepackage{amsthm}
\usepackage{fancyvrb}
\usepackage{fancyref}
%\usepackage[a4paper]{geometry}
\usepackage{graphicx}
\usepackage{hyperref}
%\usepackage{theoremref}
\usepackage{datatool}
\usepackage{float}
\usepackage[framemethod=tikz]{mdframed}
\usepackage{listingsutf8}
\usepackage{enumerate}
\usepackage{comment}
\usepackage{epstopdf}
\usepackage{caption}
\usepackage{subcaption}
%\usepackage{titling}
\usepackage{pgf,tikz}
\usepackage{pstricks-add}
%\usepackage[shortlabels]{enumitem}
\usepackage{mathtools}
\usepackage{tabu}

\usepackage{accents}
\usetikzlibrary{shapes}
\usetikzlibrary{arrows}

\setlength{\parskip}{8pt plus 1pt minus 1pt}

% Verdur ad vera her, sumir pakkar dependa a thetta.
\usepackage[icelandic]{babel}

%\newcommand{\subtitle}[1]{%
%  \posttitle{%
%    \par\end{center}
%    \begin{center}\large#1\end{center}
%    \vskip0.5em}%
%}
%viljum ekki númeraða kafla á dæmum



\newcommand{\nonums}{\setcounter{secnumdepth}{-1}}

%flýtiskipanir
\newcommand{\e}{\emph}


%\newcommand{\R}{\Real}
%\newcommand{\C}{\Complex}
%\newcommand{\Z}{\Integer}
%\newcommand{\N}{\Natural}
%\newcommand{\Q}{\Rational}
\newcommand{\R}{\mathbb{R}}
\newcommand{\X}{\mathbb{X}}
\newcommand{\Y}{\mathbb{Y}}
\newcommand{\K}{\mathbb{K}}
\newcommand{\C}{\mathbb{C}}
\newcommand{\Con}{\mathcal{C}}
\newcommand{\Z}{\mathbb{Z}}
\newcommand{\N}{\mathbb{N}}
\newcommand{\Q}{\mathbb{Q}}
\newcommand{\f}{\frac}
\newcommand{\1}{\frac{1}}
\newcommand{\eps}{\f{\epsilon}}
\newcommand{\Lra}{\Leftrightarrow}
\newcommand{\Th}{\text{ þegar }}
\newcommand{\Ef}{\text{ ef }}
\newcommand{\Og}{\text{ og }}

\newcommand*\circled[1]{\tikz[baseline=(char.base)]{
    \node[shape=circle,draw,inner sep=1pt] (char) {#1};}}

\newcommand{\hri}[1]{\text{\circled{\(#1\)}}}
\newcommand{\lin}[2]{\text{\(#1\)-\(#2\)}}

\newcommand{\inner}[1]{\accentset{\circ}{#1}}
\newcommand{\eR}{\widetilde{\R}}

\newcommand{\sumninfty}[1]{\sum_{n = {#1}}^{\infty}}
\newcommand{\sumoinfty}{\sum_{n = 1}^{\infty}}
\newcommand{\summinfty}{\sum_{m = 1}^{\infty}}
\newcommand{\sumzinfty}{\sum_{n = 0}^{\infty}}

\newcommand{\com}[1]{\set{\text{#1}}}
\newcommand{\Com}[1]{\set{\text{Athsmd: \text{#1}}}}

\newcommand{\ub}[2]{\underbrace{#1}_{\text{#2}}}
\newcommand{\ubt}[2]{$\ub{\text{#1}}{#2}$}


\newenvironment{inum}{\begin{enumerate}[label=(\roman*).]}{\end{enumerate}}
\newenvironment{anum}{\begin{enumerate}[label=(\alph*).]}{\end{enumerate}}


\newcommand{\bcondef}{\left\{ \begin{array}{l l}}
\newcommand{\abs}[1]{\left|#1\right|}

\newcommand{\econdef}{\end{array} \right.}
\DeclarePairedDelimiter{\condef}{\bcondef}{\econdef}

\DeclarePairedDelimiter{\ceil}{\lceil}{\rceil}
\DeclarePairedDelimiter{\floor}{\lfloor}{\rfloor}
\DeclarePairedDelimiter{\set}{\{}{\}}
\DeclarePairedDelimiter{\braket}{\langle}{\rangle}

\newcommand{\sep}{\;|\;}

\newcommand{\fig}[2]{
\begin{figure}[H]
  \centering
  \includegraphics{#1}
  \caption{#2}
  \label{fig:#1}
\end{figure}
}

\newcommand{\incsteiner}[1]{\ifglaerur \input{GeoGebra/Steiner/Beamer/#1} \else \input{GeoGebra/Steiner/#1} \fi}


\renewcommand{\qedsymbol}{\textbf{Q.E.D}}

\mdfdefinestyle{theoremstyle}{%
roundcorner=5pt, %
%leftmargin=1pt, rightmargin=1pt, %
hidealllines=true, %
align = center, % align theenvironment itself (left, center, rigth)
nobreak=true,
}

\numberwithin{equation}{section}
\theoremstyle{plain}
\newtheorem*{setn}{Setning}
\newtheorem*{fylgisetn}{Fylgisetning}
\newtheorem{setn*}[equation]{Setning}
\newtheorem*{hsetn}{Hjálparsetning}
\newtheorem{smid}{\textbf{Smíð}}
\newtheorem*{fylgismid}{\textbf{Fylgismíð}}
\newtheorem{hfsmid}{\textbf{Hringfarasmíð}}
\newtheorem{hogrsmid}{\textbf{Reglustiku og fast hringssmíð}}

\theoremstyle{definition}
\newtheorem*{skgr}{Skilgreining}
\newtheorem{skgr*}[equation]{Skilgreining}
\newtheorem*{daemi}{Dæmi}
\newtheorem*{frumsenda}{Frumsenda}
\newtheorem*{lausn}{Lausn}


\theoremstyle{remark}
\newtheorem*{ath}{Athugasemd}
\newtheorem*{innsk}{Innskot}




\lstset{  literate={á}{{\'a}}1
                  {ó}{{\'o}}1
                  {ú}{{\'u}}1
                  {ð}{{\dh}}1
                  {í}{{\'i}}1
                  {é}{{\'e}}1
                  {ö}{{\"o}}1
                  {þ}{{\th}}1
                  {æ}{{\ae}}1
                  {ý}{{\'y}}1
                  {Á}{{\'A}}1
                  {Ó}{{\'O}}1
                  {Ú}{{\'U}}1
                  {Ð}{{\DH}}1
                  {Í}{{\'I}}1
                  {É}{{\'E}}1
                  {Ö}{{\"O}}1
                  {Þ}{{\TH}}1
                  {Æ}{{\AE}}1
                  {Ý}{{\'Y}}1}
